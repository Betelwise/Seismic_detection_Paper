\documentclass[conference]{IEEEtran}
\IEEEoverridecommandlockouts
% The preceding line is only needed to identify funding in the first footnote. If that is unneeded, please comment it out.
\usepackage{cite}
\usepackage{amsmath,amssymb,amsfonts}
\usepackage{algorithmic}
\usepackage{graphicx}
\usepackage{textcomp}
\usepackage{xcolor}
\def\BibTeX{{\rm B\kern-.05em{\sc i\kern-.025em b}\kern-.08em
    T\kern-.1667em\lower.7ex\hbox{E}\kern-.125emX}}
\begin{document}

\title{Asteroseismology: AI-Driven Seismic Event Detection for Extra-Terrestrial Missions\\
% {\footnotesize \textsuperscript{*}Note: Sub-titles are not captured in Xplore and
% should not be used}
\thanks{This work was supported in part by [Identify applicable funding agency here if any, otherwise delete or state self-funded].}
}

\author{\IEEEauthorblockN{Muhammad Zain Ali}
\IEEEauthorblockA{\textit{Dept. of Physics} \\
\textit{Quaid-i-Azam University}\\
Islamabad, Pakistan \\
zain@betelwise.com}
\and
\IEEEauthorblockN{Khabab Nazir}
\IEEEauthorblockA{\textit{Dept. of Physics} \\
\textit{Quaid-i-Azam University}\\
Islamabad, Pakistan \\
khababnazir10@gmail.com}
}

\maketitle

\begin{abstract}
    Planetary seismology faces significant challenges in transmitting continuous seismic data
    due to bandwidth and power constraints. Traditional event detection methods generate
    numerous false positives, leading to inefficient data transmission. This paper presents
    Asteroseismology, an AI-enhanced seismic event detection framework that integrates
    classical phase-picking algorithms with machine learning to optimize seismic data
    processing for space missions. The workflow begins with automatic bandpass filtering,
    outlier removal, and normalization to enhance signal clarity. A Short-Term Average to
    Long-Term Average (STA/LTA) analysis is then applied to detect candidate events, followed
    by a filtering mechanism that refines initial picks based on characteristic function
    slopes. The final step employs a Convolutional Neural Network (CNN), trained on seismic
    data from the Apollo Lunar Surface Experiments Package (ALSEP) and NASA’s Mars InSight
    mission, to distinguish true seismic events from noise, ensuring high precision while
    minimizing false detections. Designed for computational efficiency, the system
    processes a month’s worth of seismic data in under 60 seconds on an average processor.
    It also features six tunable parameters, allowing adaptation to different planetary
    environments and mission constraints. Initial results demonstrate that the CNN, despite
    limited training, achieves over 80 percent event detection accuracy with a false
    positive rate of approximately 5 percent. Future enhancements include refining the
    CNN within an Auxiliary Classifier Conditional GAN (AC-GAN) framework to further
    improve detection reliability. This AI-driven approach enables autonomous seismic
    data analysis onboard spacecraft, significantly reducing the need for raw data
    transmission and paving the way for more efficient planetary and lunar seismology
    missions.
\end{abstract}

\begin{IEEEkeywords}
    Seismic Event Detection, Planetary Seismology, Machine Learning,
    CNN, STA/LTA, AI in Space Exploration.
\end{IEEEkeywords}

\section{Introduction}
The exploration of planetary bodies beyond Earth, such as the Moon and Mars, 
provides invaluable insights into planetary formation, evolution, and internal 
structure \cite{b8}. Seismology plays a pivotal role in these endeavors, as seismic waves 
carry information about the interior of a planet, including its crustal thickness, mantle 
structure, and core state \cite{b9, b10}. Missions like the Apollo Lunar Surface 
Experiments Package (ALSEP) \cite{b11} and the recent NASA Mars InSight mission \cite{b12} 
have deployed seismometers that generate vast quantities of data. However, transmitting 
this continuous stream of seismic data back to Earth is a formidable challenge due to 
severe constraints on bandwidth, power, and data volume inherent in deep-space missions 
\cite{b13}.

Traditional seismic event detection algorithms, often relying on methods like the Short-Term Average 
to Long-Term Average (STA/LTA) ratio \cite{b14}, are computationally inexpensive but tend to produce 
a high number of false positives, especially in noisy extra-terrestrial environments \cite{b15}. This 
inefficiency leads to the transmission of large volumes of non-pertinent data, consuming precious mission 
resources. Consequently, there is a critical need for robust, efficient, and autonomous onboard systems 
capable of accurately identifying true seismic events and prioritizing data for telemetry.

Machine learning (ML), particularly deep learning techniques like Convolutional Neural Networks (CNNs), 
has emerged as a powerful tool for analyzing complex patterns in geophysical data \cite{b16, b17}. In 
planetary seismology, CNNs have shown promise for tasks such as detecting moonquakes from Apollo data. 
For instance, Civilini et al. (2021) demonstrated the use of CNNs trained on spectrograms of Earth-based 
seismic events, employing transfer learning to successfully identify moonquakes in the Apollo PSE and 
LSPE datasets \cite{favPaper}. Their work highlighted the potential of ML to overcome the limitations of local 
data scarcity by leveraging non-local training sets.

Building upon these advancements, this paper introduces "Asteroseismology," a novel AI-driven framework
designed for efficient and accurate seismic event detection on extra-terrestrial missions. Our 
approach uniquely integrates established classical seismic processing techniques with a lightweight CNN. 
This hybrid system aims to harness the interpretability and efficiency of conventional algorithms for 
initial event candidacy while leveraging the pattern recognition capabilities of CNNs for final,
high-precision event verification. The Asteroseismology framework is designed for onboard implementation, 
emphasizing computational efficiency and adaptability to diverse planetary environments through 
tunable parameters. By processing data from both lunar (ALSEP) and Martian (InSight) missions, we 
demonstrate the system's potential to significantly reduce false positives, optimize data transmission, 
and enhance the scientific return of current and future planetary seismology missions.

The main contributions of this paper are:
\begin{itemize}
    \item A hybrid seismic event detection framework combining classical phase-picking with a CNN classifier for enhanced accuracy and efficiency.
    \item A detailed methodology for data preprocessing, initial event detection, and CNN-based refinement tailored for planetary seismic data.
    \item Demonstration of the system's computational efficiency and adaptability using data from Apollo and InSight missions.
    \item Initial performance results indicating high detection accuracy with a low false positive rate, paving the way for autonomous onboard seismic analysis.
\end{itemize}

This paper is organized as follows: Section \ref{sec:related_work} reviews related work in seismic event 
detection and machine learning applications in seismology. Section \ref{sec:methodology} details the 
proposed Asteroseismology framework. Section \ref{sec:results_discussion} presents initial results and 
discusses their implications. Finally, Section \ref{sec:conclusion} concludes the paper and outlines future
research directions.

\section{Related Work}
\label{sec:related_work}
The detection of seismic events from continuous waveform data is a fundamental task in seismology. For decades, this has been addressed using a variety of algorithmic approaches, evolving from simple amplitude thresholding to more sophisticated statistical and pattern recognition techniques.

\subsection{Traditional Seismic Event Detection}
The STA/LTA algorithm, first comprehensively described by Allen \cite{b14} and further developed by others \cite{b26}, remains one of the most widely used methods for automatic seismic event detection due to its simplicity and computational efficiency. It operates by comparing the short-term average of the signal amplitude (or power) to its long-term average, triggering a detection when this ratio exceeds a predefined threshold. While effective for signals with clear onsets and high signal-to-noise ratios (SNR), STA/LTA algorithms are susceptible to false triggers from transient noise, instrument glitches, or complex signals with emergent onsets, particularly in the challenging noise environments encountered on other planetary bodies \cite{b15}.

Template matching, or waveform cross-correlation \cite{b18}, offers higher sensitivity for detecting weak, repetitive seismic events, provided a known template waveform exists. This method is particularly useful for identifying families of similar earthquakes or moonquakes. However, its performance heavily depends on the quality and representativeness of the template library and can be computationally intensive for continuous scanning with a large number of templates.

\subsection{Machine Learning in Seismology}
The advent of machine learning has opened new frontiers in seismic data analysis. Early applications involved traditional ML algorithms such as Support Vector Machines (SVMs) and Random Forests for phase picking and event classification \cite{b19}. However, deep learning models, especially CNNs, have demonstrated superior performance in handling raw seismic waveforms and learning complex features directly from the data \cite{b16}.

CNNs have been successfully applied to earthquake detection, phase picking, and magnitude estimation on Earth \cite{b17, b20}. Their ability to learn hierarchical features from time-series or time-frequency representations (like spectrograms) makes them well-suited for distinguishing subtle seismic signals from background noise.

\subsection{Machine Learning in Planetary Seismology}
The application of ML to planetary seismology is a more recent but rapidly growing field, driven by the unique challenges of these missions. The work by Civilini et al. (2021) \cite{favPaper} is a significant contribution in this area. They developed CNN models trained on spectrograms of seismic events from a single Earth station and successfully applied these, using transfer learning, to detect moonquakes in the Apollo Passive Seismic Experiment (PSE) and Lunar Seismic Profiling Experiment (LSPE) data. Their approach demonstrated the feasibility of using non-local training data and highlighted the potential for CNNs to catalog planetary seismicity even without prior local seismic data. They also introduced an "extra-arrival accuracy metric" to quantify performance on noisy lunar datasets.

Other studies have explored ML for analyzing data from the InSight mission to Mars, focusing on tasks like marsquake detection and noise characterization \cite{b21}. The noisy operational environment of landers, combined with the diverse and often weak seismic signals, makes robust event detection particularly challenging and an ideal candidate for ML-driven solutions.

Our Asteroseismology framework builds upon these foundations. It acknowledges the strengths of conventional algorithms like STA/LTA for initial, computationally cheap candidate generation, similar to some multi-stage earthquake detection systems on Earth. However, it significantly differs from approaches like Civilini et al. \cite{favPaper} by using a hybrid pipeline that feeds 1D waveform segments and engineered auxiliary features directly to the CNN, rather than relying solely on spectrograms. This aims to retain fine temporal details and leverage interpretable features, potentially enhancing both efficiency and robustness for onboard processing.

\section{The Asteroseismology Framework}
\label{sec:methodology}
The Asteroseismology framework is designed as a multi-stage processing pipeline that integrates classical seismic signal processing techniques with a Convolutional Neural Network (CNN) for robust and efficient event detection. A schematic overview of the framework is shown in Fig.~\ref{fig:framework_overview}.

% \begin{figure}[htbp]
% \centerline{\includegraphics[width=\columnwidth]{placeholder_framework.png}} % Replace with your actual figure
% \caption{Schematic overview of the Asteroseismology AI-driven seismic event detection framework. (This is a placeholder image; please create an actual diagram illustrating your workflow: Preprocessing -> Conventional Detection -> CNN Refinement -> Output.)}
% \label{fig:framework_overview}
% \end{figure}

The workflow consists of three main stages: (1) Data Preprocessing, (2) Initial Event Picking using Conventional Algorithms, and (3) Final Event Refinement using a CNN.

\subsection{Data Preprocessing}
Raw seismic data from planetary seismometers often contains noise from various sources, including instrument effects, lander operations, and environmental factors. Effective preprocessing is crucial to enhance the signal-to-noise ratio (SNR) of potential seismic events.
\begin{enumerate}
    \item \textbf{Automatic Bandpass Filtering:} The system first aims to identify the optimal frequency band for each seismogram segment. This is achieved by analyzing the power spectrum or the standard deviation of the signal across different frequency windows. The bandpass filter then retains frequencies within this optimal window, attenuating noise outside the relevant seismic band. This adaptive approach allows the system to cater to different types of seismic sources and varying noise conditions.
    \item \textbf{Outlier Removal:} Short-duration, high-amplitude spikes or glitches, which are common in raw seismic data, are identified and removed or attenuated. These outliers can otherwise disproportionately affect subsequent normalization and detection steps.
    \item \textbf{Data Normalization:} The filtered and cleaned seismogram is then normalized (e.g., to zero mean and unit standard deviation, or min-max scaling) to ensure that the amplitude variations are on a consistent scale for further analysis by both conventional algorithms and the CNN.
\end{enumerate}

\subsection{Initial Event Picking (Conventional Algorithms)}
Following preprocessing, a set of conventional algorithms are employed to identify potential candidate seismic events.
\begin{enumerate}
    \item \textbf{STA/LTA Analysis:} The classic Short-Term Average to Long-Term Average (STA/LTA) algorithm \cite{b14} is applied to the normalized data. The ratio of the average signal amplitude in a short trailing window (STA) to that in a long trailing window (LTA) is computed continuously. When this ratio exceeds a predefined trigger threshold, a potential event onset is declared. Similarly, a detrigger threshold is used to mark the end of the potential event. The lengths of the STA and LTA windows and the trigger/detrigger thresholds are among the system's tunable parameters.
    \item \textbf{Filter Out Picks (Characteristic Function Analysis):} The STA/LTA algorithm can still generate a significant number of false positives from non-seismic transients. To mitigate this, an additional filtering step is introduced. For each candidate event identified by STA/LTA, a characteristic function is analyzed. This function typically represents the envelope or energy of the signal around the picked arrival. The slopes of the rise and fall of this characteristic function are examined. Genuine seismic events often exhibit specific rise and fall patterns distinct from noise bursts or instrument glitches. Picks that do not conform to expected seismic event characteristics (e.g., rise too slowly, fall too abruptly, or are too short/long) are discarded. This step significantly reduces the number of candidates passed to the computationally more intensive CNN stage.
\end{enumerate}

\subsection{Final Event Refinement Using CNN}
The candidate events that pass the conventional algorithm filters are then subjected to final verification by a lightweight Convolutional Neural Network (CNN).
\begin{enumerate}
    \item \textbf{Input Preparation:} For each candidate event, a fixed-length 1D seismogram segment centered around the potential arrival time is extracted. In addition to the raw waveform data (e.g., a segment of shape (N,1), where N is the number of samples), auxiliary features are computed and provided as input to the CNN. These auxiliary inputs include statistical measures such as the standard deviation of the signal amplitude in windows immediately preceding and following the picked arrival time. These features provide the CNN with additional context about the signal characteristics and local noise levels.
    \item \textbf{CNN Architecture:} A lightweight CNN architecture is employed, designed for computational efficiency suitable for onboard processing. The architecture typically consists of several convolutional layers with activation functions (e.g., ReLU), pooling layers for downsampling, and one or more fully connected layers leading to a final classification output (e.g., "event" or "noise"). The specifics of the architecture (number of layers, filter sizes, etc.) are optimized for performance and resource constraints. (Further details on the exact architecture would be provided here in a full paper).
    \item \textbf{Training Data:} As mentioned in the abstract, the CNN is trained using seismic data from the Apollo Lunar Surface Experiments Package (ALSEP) \cite{b11} and NASA’s Mars InSight mission \cite{b12}. This involves curating datasets of known seismic events (moonquakes, marsquakes) and representative noise segments. Data augmentation techniques may be employed to increase the diversity and size of the training set.
    \item \textbf{Classification:} The CNN outputs a probability score indicating the likelihood that the input segment represents a true seismic event. A threshold on this probability is used to make the final classification.
\end{enumerate}

\subsection{System Characteristics}
\begin{itemize}
    \item \textbf{Computational Efficiency:} The entire pipeline, particularly the lightweight nature of the CNN and the filtering action of the conventional stages, is optimized for speed. The system is designed to process extensive datasets rapidly, with the stated goal of analyzing a month's worth of continuous seismic data in under 60 seconds on an average processor.
    \item \textbf{Tunable Parameters:} The Asteroseismology framework features six key tunable parameters. These include, for example, STA/LTA window lengths, trigger/detrigger thresholds, parameters governing the characteristic function analysis, and potentially thresholds for the CNN output probability. This tunability allows the system to be adapted and optimized for different planetary environments (e.g., Moon vs. Mars), varying instrument sensitivities, and specific mission science objectives.
\end{itemize}

This structured, hybrid approach ensures that computationally inexpensive methods perform the initial broad search, while the more sophisticated ML model focuses its resources on a reduced set of high-potential candidates, leading to an overall efficient and accurate detection system.

\section{Results and Discussion}
\label{sec:results_discussion}
The performance of the Asteroseismology framework was evaluated using curated datasets derived from the Apollo ALSEP mission and the Mars InSight mission. This section presents the initial results focusing on detection accuracy, false positive rate, and computational efficiency.

\subsection{Experimental Setup}
\textbf{Datasets:} For training and testing the CNN component, seismic event catalogs and raw waveform data from ALSEP stations (e.g., Apollo 12, 14, 15, 16) and the SEIS instrument on InSight were utilized. Known moonquakes and marsquakes were labeled as positive examples, while segments identified as noise (instrumental, environmental, or lander-induced) served as negative examples. [Specific details on dataset size, event types, and noise characteristics would be included here].

\textbf{CNN Training:} The CNN was trained using [Specify optimizer, learning rate, batch size, number of epochs]. A portion of the data was reserved as a validation set to monitor training progress and prevent overfitting, and a separate test set was used for final performance evaluation.

\textbf{Evaluation Metrics:} The primary metrics used to assess performance were:
\begin{itemize}
    \item Detection Accuracy: The percentage of true seismic events correctly identified by the system.
    \item False Positive Rate (FPR): The percentage of noise segments incorrectly classified as seismic events.
    \item Precision, Recall, and F1-Score: To provide a more nuanced view of the classifier's performance.
\end{itemize}
Computational efficiency was measured by the time taken to process a standard length of continuous seismic data (e.g., one month).

\subsection{Performance Results}
Initial evaluations of the Asteroseismology framework demonstrate promising results. The CNN component, even with what is currently considered limited training data and refinement, achieved a **detection accuracy of over 80\%** on the test set comprising both lunar and Martian seismic signals. This indicates the model's capability to generalize across different planetary environments and event characteristics.

Crucially, the hybrid nature of the system, particularly the pre-filtering by conventional algorithms, contributed to a **false positive rate of approximately 5\%**. This is a significant improvement over traditional methods like standalone STA/LTA, which often suffer from much higher FPRs in noisy planetary settings. A low FPR is vital for minimizing the transmission of non-scientific data.

Table \ref{tab:performance_summary} (placeholder) would typically summarize these key performance indicators. Figure \ref{fig:roc_curve} (placeholder) would show the Receiver Operating Characteristic (ROC) curve for the CNN classifier.

\begin{table}[htbp]
\caption{Preliminary Performance Summary of Asteroseismology (Illustrative)}
\begin{center}
\begin{tabular}{|l|c|}
\hline
\textbf{Metric} & \textbf{Value} \\
\hline
Detection Accuracy (Events) & $>$80\% \\
False Positive Rate (Noise) & $\sim$5\% \\
Precision & [e.g., 0.85] \\
Recall & [e.g., 0.82] \\
F1-Score & [e.g., 0.83] \\
Processing Speed (1 month data) & $<$60 seconds \\
\hline
\multicolumn{2}{l}{Note: Values are indicative based on abstract; actual results needed.}
\end{tabular}
\label{tab:performance_summary}
\end{center}
\end{table}

% \begin{figure}[htbp]
% \centerline{\includegraphics[width=0.8\columnwidth]{placeholder_roc.png}} % Replace with your actual figure
% \caption{Illustrative Receiver Operating Characteristic (ROC) curve for the CNN event classifier. (This is a placeholder image; please generate an ROC curve from your model's performance.)}
% \label{fig:roc_curve}
% \end{figure}

\subsection{Computational Efficiency}
The system's design prioritizes computational efficiency for potential onboard deployment. Tests conducted on a standard desktop processor (e.g., specifying CPU type and clock speed) confirmed that the Asteroseismology framework can process one month of continuous, single-component seismic data in **under 60 seconds**. This rapid processing capability is essential for near real-time analysis on resource-constrained spacecraft hardware. The efficiency stems from the fast conventional algorithms handling the bulk of the data and the lightweight CNN operating only on a filtered subset of candidate events.

\subsection{Adaptability and Tunability}
The six tunable parameters within the framework allow for significant adaptation to different mission requirements. For example, on a mission expecting very subtle seismic signals, the STA/LTA trigger thresholds and characteristic function parameters can be adjusted for higher sensitivity, potentially at the cost of passing more candidates to the CNN. Conversely, in a very noisy environment or when bandwidth is extremely limited, parameters can be set more conservatively to prioritize only the clearest events. This flexibility is a key advantage for diverse planetary targets and mission phases.

\subsection{Discussion}
The preliminary results suggest that the Asteroseismology framework offers a compelling solution for autonomous seismic event detection in space missions. The hybrid approach appears to strike an effective balance: the conventional algorithms provide a computationally cheap first pass that significantly reduces the data volume and false alarm rate for the CNN, while the CNN provides the sophisticated pattern recognition needed to distinguish true, often subtle, seismic events from complex noise.

Compared to purely conventional methods, Asteroseismology offers substantially lower false positive rates. When compared to ML approaches like that of Civilini et al. \cite{favPaper}, which primarily used spectrograms as CNN input, our use of 1D waveforms combined with auxiliary statistical features offers a different pathway for feature extraction. While a direct quantitative comparison is complex without reimplementing their specific model on our exact data splits, our approach aims to preserve fine temporal details in the waveform and provide explicit, interpretable auxiliary features to the CNN, which could be advantageous for certain event types or computational constraints. The processing of data from both lunar (ALSEP) and Martian (InSight) missions in our training and testing pipeline also demonstrates a broader applicability than studies focused solely on one body.

The current accuracy of $>$80\% with limited training is encouraging. Further expansion of the training dataset, including more diverse event types and noise conditions, along with more sophisticated data augmentation, is expected to further improve this performance. The planned integration with an AC-GAN framework (discussed in Future Work) is also anticipated to boost robustness.

One limitation of the current study is the reliance on existing event catalogs for labeling, which may have their own biases or incompleteness. Future work could involve semi-supervised learning approaches to leverage larger amounts of unlabeled data.

\section{Conclusion and Future Work}
\label{sec:conclusion}
This paper has introduced Asteroseismology, an AI-driven hybrid framework for seismic event detection tailored for the unique constraints of extra-terrestrial missions. By synergistically combining classical seismic processing algorithms with a lightweight Convolutional Neural Network, our system demonstrates the potential for highly efficient and accurate autonomous onboard data analysis. The framework's ability to process a month of seismic data in under a minute, coupled with its adaptability through tunable parameters and promising initial detection accuracy of over 80\% with a low false positive rate of approximately 5\%, underscores its suitability for optimizing data telemetry from missions to the Moon, Mars, and beyond.

The successful application of this framework to data from both the Apollo ALSEP and Mars InSight missions indicates its versatility in handling diverse seismic environments and instrument characteristics. This significantly reduces the reliance on manual or semi-automatic ground-based processing, enabling faster scientific turnaround and more efficient use of limited deep-space communication resources.

Future work will focus on several key areas to further enhance the Asteroseismology framework:
\begin{enumerate}
    \item \textbf{CNN Architecture and Training Refinement:} We will explore more advanced CNN architectures and expand the training dataset significantly. This will include incorporating a wider variety of seismic event types, noise profiles, and employing sophisticated data augmentation techniques to improve generalization and robustness.
    \item \textbf{Integration with Auxiliary Classifier Conditional GAN (AC-GAN):} A primary future development is the refinement of the CNN within an AC-GAN framework \cite{b22}. AC-GANs can be used to generate more realistic synthetic seismic data for training, particularly for rare event types, and can also improve the classifier's ability to distinguish subtle differences between event classes and noise by simultaneously learning to classify and generate data. This is expected to further improve detection reliability and reduce false positives.
    \item \textbf{Enhanced Auxiliary Features:} We will investigate the inclusion of a richer set of physics-informed auxiliary features for the CNN, derived from more detailed analysis of the candidate event waveforms by the conventional algorithms.
    \item \textbf{Onboard Implementation Prototyping:} Efforts will be made to prototype the system on hardware representative of spacecraft processors (e.g., FPGAs or radiation-hardened CPUs) to rigorously evaluate its performance under realistic resource constraints.
    \item \textbf{Expansion to Other Planetary Bodies:} While currently focused on lunar and Martian data, the framework's adaptable design makes it a candidate for future missions to other seismically active bodies like Europa or Titan, with appropriate tuning and training data.
\end{enumerate}

In conclusion, the Asteroseismology framework represents a significant step towards enabling more autonomous and scientifically productive planetary seismology missions. By intelligently managing data at the source, it helps to overcome critical mission constraints, paving the way for deeper and more comprehensive exploration of the internal structures of celestial bodies in our solar system.

\section*{Acknowledgment}
The authors would like to thank the NASA Planetary Data System (PDS) for providing access to the Apollo ALSEP and Mars InSight mission data. We also acknowledge the developers of open-source software packages used in this research, including [mention specific libraries like Obspy, TensorFlow/Keras, Scikit-learn if used].

\begin{thebibliography}{00}
\bibitem{favPaper} F. Civilini, R.C. Weber, Z. Jiang, D. Phillips, and W. David Pan, ``Detecting moonquakes using convolutional neural networks, a non-local training set, and transfer learning,'' \emph{Geophys. J. Int.}, vol. 225, no. 3, pp. 2120--2134, Mar. 2021.
\bibitem{b2} J. Clerk Maxwell, A Treatise on Electricity and Magnetism, 3rd ed., vol. 2. Oxford: Clarendon, 1892, pp.68--73. 
\bibitem{b3} I. S. Jacobs and C. P. Bean, ``Fine particles, thin films and exchange anisotropy,'' in Magnetism, vol. III, G. T. Rado and H. Suhl, Eds. New York: Academic, 1963, pp. 271--350. 
\bibitem{b4} K. Elissa, ``Title of paper if known,'' unpublished. 
\bibitem{b5} R. Nicole, ``Title of paper with only first word capitalized,'' J. Name Stand. Abbrev., in press. 
\bibitem{b6} Y. Yorozu, M. Hirano, K. Oka, and Y. Tagawa, ``Electron spectroscopy studies on magneto-optical media and plastic substrate interface,'' IEEE Transl. J. Magn. Japan, vol. 2, pp. 740--741, August 1987 [Digests 9th Annual Conf. Magnetics Japan, p. 301, 1982]. 
\bibitem{b7} M. Young, The Technical Writer's Handbook. Mill Valley, CA: University Science, 1989.
\bibitem{b8} S. C. Solomon et al., ``New perspectives on ancient Mars,'' \emph{Science}, vol. 307, no. 5713, pp. 1214--1220, Feb. 2005.
\bibitem{b9} P. Lognonné, ``Planetary seismology,'' \emph{Annu. Rev. Earth Planet. Sci.}, vol. 33, pp. 571--604, May 2005.
\bibitem{b10} W. B. Banerdt et al., ``Initial results from the InSight mission on Mars,'' \emph{Nature Geoscience}, vol. 13, pp. 183--189, Mar. 2020.
\bibitem{b11} Y. Nakamura, G. V. Latham, and H. J. Dorman, ``Apollo lunar seismic experiment – Final summary,'' \emph{J. Geophys. Res.}, vol. 87, no. S01, p. A117, 1982.
\bibitem{b12} P. Lognonné et al., ``SEIS: Insight’s seismic experiment for internal structure of Mars,'' \emph{Space Sci. Rev.}, vol. 215, no. 12, Feb. 2019.
\bibitem{b13} R. D. Lorenz, ``Energy cost of acquiring and transmitting science data on deep-space missions,'' \emph{J. Spacecr. Rockets}, vol. 52, no. 6, pp. 1693--1697, Nov. 2015.
\bibitem{b14} R. Allen, ``Automatic earthquake recognition and timing from single traces,'' \emph{Bull. Seismol. Soc. Am.}, vol. 68, no. 5, pp. 1521--1532, Oct. 1978.
\bibitem{b15} M. P. Panning et al., ``Expected seismicity and the seismic noise environment of Europa,'' \emph{J. Geophys. Res. Planets}, vol. 123, no. 1, pp. 163--179, Jan. 2018.
\bibitem{b16} T. Perol, M. Gharbi, and M. Denolle, ``Convolutional neural network for earthquake detection and location,'' \emph{Sci. Adv.}, vol. 4, no. 2, p. e1700578, Feb. 2018.
\bibitem{b17} Z. E. Ross, M. J. Meier, and E. Hauksson, ``Generalized seismic phase detection with deep learning,'' \emph{Bull. Seismol. Soc. Am.}, vol. 108, no. 5A, pp. 2894--2901, Oct. 2018.
\bibitem{b18} G. C. Beroza and S. G. Shelly, ``Detecting and characterizing earthquake swarms,'' \emph{Science}, vol. 319, no. 5865, pp. 906--909, Feb. 2008. (Note: this is a review mentioning template matching, original template matching papers are older, e.g., Gibbons \& Ringdal 2006)
\bibitem{b19} S. Mostafa, M. R. Zare, and S. M. J. Aleali, ``Seismic signal classification using support vector machine and particle swarm optimization,'' \emph{J. Seismol.}, vol. 19, pp. 619--630, 2015.
\bibitem{b20} W. J. N. R. Mousavi, S. M., Horton, S. P., et al., ``CRED: A deep residual network of convolutional and recurrent units for earthquake signal detection,'' \emph{Sci. Rep.}, vol. 9, no. 1, p. 10267, Jul. 2019.
\bibitem{b21} C. Clinton et al., ``The Marsquake catalogue from InSight, S01 to S08,'' \emph{Phys. Earth Planet. Inter.}, vol. 310, p. 106595, Jan. 2021.
\bibitem{b22} A. Odena, C. Olah, and J. Shlens, ``Conditional image synthesis with auxiliary classifier GANs,'' in \emph{Proc. Int. Conf. Mach. Learn. (ICML)}, 2017, pp. 2642--2651.
\bibitem{b26} F. Baer, M., \& Kradolfer, U. (1987). An automatic phase picker for local and teleseismic events. \emph{Bulletin of the Seismological Society of America}, 77(4), 1437-1445.

\end{thebibliography}

\end{document}