\documentclass[11pt,a4paper]{article}

\usepackage[utf8]{inputenc}
\usepackage{amsmath}
\usepackage{graphicx}
\usepackage[numbers,sort&compress]{natbib} % For numbered, sorted, compressed citations
\usepackage{hyperref}
\usepackage{geometry}
\usepackage{times} % Using Times font
\usepackage{authblk} % For author affiliations
\usepackage{booktabs} % For better tables
\usepackage{caption} % For captions
\usepackage{subcaption} % For subfigures
\usepackage{float} % For better figure placement

\geometry{a4paper, margin=1in}

\title{\textbf{Betelwise1: A Hybrid Conventional-CNN Approach for Efficient Planetary Seismic Event Detection and Analysis}}
\author[1]{[Your Name/Team Name]} % Replace with your name/team
\affil[1]{[Your Affiliation/University]} % Replace with your affiliation
\date{\today}

% --- Placeholder for Keywords ---
% \keywords{Planetary Seismology, Machine Learning, Convolutional Neural Network, Seismic Event Detection, Mars, Moon, STA/LTA, Data Optimization}

\begin{document}

\maketitle

\begin{abstract}
This project, Betelwise1, is centered around developing an efficient system to detect and analyze seismic events on planetary bodies like Mars or the Moon. By combining traditional phase-picking techniques with modern machine learning, the system is designed to optimize data transmission from distant space missions while ensuring high accuracy and reliability. This system blends tunable conventional algorithms with a Convolutional Neural Network (CNN) to provide an end-to-end seismic event detection and analysis workflow. The conventional algorithms handle the initial stages of noise filtering and phase-picking, while the CNN is responsible for the final refinement, ensuring that only the most relevant seismic events are retained. Initial Picking (Conventional Algorithms) involves Adaptive Frequency Band Estimation to find the optimal bandpass filter window, application of the bandpass filter, outlier removal, data normalization, and STA/LTA analysis. A characteristic function-based filter is then applied to distinguish between events and noise. Final Event Refinement uses a lightweight CNN that takes a 1D seismogram segment (shape 5565,1) and auxiliary inputs (standard deviation values before and after the arrival time) to differentiate real seismic events from residual noise. While in early stages, this hybrid approach shows promise for robust and efficient on-board planetary seismic analysis.
\end{abstract}

\noindent\textbf{Keywords:} Planetary Seismology, Machine Learning, Convolutional Neural Network, Seismic Event Detection, Mars, Moon, STA/LTA, Data Optimization, Hybrid System.

\section{Introduction}
The exploration of planetary bodies beyond Earth offers profound insights into their internal structure, geological activity, and potential habitability. Seismology plays a pivotal role in these endeavors, providing direct measurements of ground motion caused by tectonic activity, impacts, or other internal processes \citep{lognonne2005planetary, giardini2020seismicity}. However, planetary missions, especially to distant bodies like Mars or the Moon, face severe constraints in terms of power, computational resources, and data telemetry bandwidth \citep{lorenz2015energy, lognonne2019seis}. Transmitting continuous high-resolution seismic data back to Earth is often infeasible, necessitating on-board processing to identify and prioritize scientifically valuable events for downlink.

Traditional seismic event detection algorithms, such as the Short-Term Average/Long-Term Average (STA/LTA) ratio \citep{allen1982automatic}, template matching \citep{gibbons2006detection}, and autocorrelation methods \citep{brown2008autocorrelation}, have been foundational in terrestrial seismology. While computationally efficient, STA/LTA methods can be sensitive to noise and require careful tuning, and template matching relies on pre-existing event templates which may not be available for unexplored planetary environments.

In recent years, machine learning (ML), particularly deep learning (DL) techniques like Convolutional Neural Networks (CNNs), has demonstrated remarkable success in various seismological tasks on Earth, including event detection, phase picking, and earthquake early warning \citep{perol2018convnetquake, ross2018p, ross2018generalized, meier2019reliable}. These methods can learn complex patterns directly from waveform data, often outperforming traditional approaches.

Pioneering work by \citet{civilini2021detecting} successfully applied CNNs to detect moonquakes using data from the Apollo Passive Seismic Experiment (PSE) and Lunar Seismic Profiling Experiment (LSPE). Their approach involved training a CNN on spectral images (spectrograms) derived from a single Earth seismic station and then transferring this knowledge to lunar data, demonstrating the potential of CNNs for planetary seismology even without local training data. This highlighted the capability of CNNs to catalog planetary seismicity and prioritize data for telemetry.

Building upon these advancements, this paper introduces Betelwise1, a hybrid system that synergizes the robustness of conventional seismic processing techniques with the discriminative power of a lightweight CNN. Unlike approaches that rely solely on spectrogram-based CNNs, Betelwise1 employs a multi-stage workflow:
\begin{enumerate}
    \item \textbf{Conventional Pre-processing and Initial Picking:} This stage uses adaptive filtering, outlier removal, normalization, STA/LTA analysis, and a novel characteristic function-based filter to identify potential event candidates and reduce false positives.
    \item \textbf{CNN-based Event Refinement:} A lightweight CNN then processes 1D seismic waveform segments centered on these candidates, along with auxiliary statistical features, to provide a final, high-confidence event detection.
\end{enumerate}
The Betelwise1 system is designed for efficiency and adaptability, aiming to optimize data transmission from missions to Mars or the Moon. By leveraging the strengths of both conventional algorithms for broad-stroke analysis and CNNs for nuanced refinement, we hypothesize that this hybrid approach can achieve high accuracy and reliability while minimizing computational overhead and data volume for downlink. This work explores an alternative to purely spectrogram-based inputs for CNNs in planetary seismology, focusing on direct waveform analysis augmented by engineered features.

\section{Related Work}
The challenge of detecting seismic events in noisy and often unfamiliar planetary environments has spurred research into various methodologies.

\subsection{Traditional Seismic Event Detection}
The STA/LTA algorithm \citep{allen1982automatic} remains a cornerstone for real-time seismic monitoring due to its simplicity and computational efficiency. It compares the short-term average power of a signal to its long-term average, flagging an event when this ratio exceeds a threshold. However, its performance can degrade in low signal-to-noise ratio (SNR) environments common on other planetary bodies, and it often requires mission-specific tuning \citep{lognonne2019seis}. Other techniques include waveform correlation or template matching \citep{gibbons2006detection}, which are effective if representative templates of expected signals are available, a condition not always met in initial planetary exploration phases. Autocorrelation methods \citep{brown2008autocorrelation} can detect repeating signals but may also be computationally intensive for continuous on-board application.

\subsection{Machine Learning in Earth Seismology}
The application of ML to terrestrial seismic data has burgeoned. Early work involved traditional ML algorithms, but deep learning, especially CNNs, has led to significant breakthroughs. \citet{perol2018convnetquake} used a CNN for earthquake detection and location from a single station. \citet{ross2018p, ross2018generalized} developed DeepDenoiser and PhaseNet, CNN-based models for seismic data denoising and phase picking, respectively, demonstrating superior performance over traditional methods. \citet{meier2019reliable} showed the utility of ML for reliable real-time signal/noise discrimination. These successes on Earth provide a strong foundation for applying similar techniques to extraterrestrial seismic data.

\subsection{Machine Learning in Planetary Seismology}
The application of ML to planetary seismic data is a more recent but rapidly growing field. \citet{knapmeyer2015identification} used Hidden Markov Models (HMMs) for event detection in Apollo 16 data. The work by \citet{civilini2021detecting} represents a key advancement, demonstrating that CNNs trained on Earth-based spectrograms can effectively detect moonquakes in Apollo PSE and LSPE data using transfer learning. They highlighted the potential for lander-side signal analysis for telemetry prioritization, addressing the power and data constraints. Their method focused on using 2D spectral images as input to the CNN, achieving impressive station accuracy averages. Their "extra-arrival accuracy metric" provided a nuanced way to evaluate performance beyond simple detection rates.

Our work, Betelwise1, builds on this context. While \citet{civilini2021detecting} effectively used spectrograms, we explore a hybrid approach that integrates established conventional techniques as a robust pre-filter for a CNN that operates on 1D time-series waveforms and auxiliary statistical features. This aims to potentially reduce the computational load of spectrogram generation on-board and leverage the interpretability of conventional pickers, while still benefiting from the CNN's pattern recognition capabilities for final refinement.

\section{The Betelwise1 System Architecture}
The Betelwise1 system is an end-to-end workflow designed for efficient and accurate seismic event detection on planetary landers. It comprises two main stages: (1) Conventional Pre-processing and Initial Picking, and (2) CNN-based Event Refinement. A conceptual overview is shown in Figure \ref{fig:system_overview}.

\begin{figure}[H]
    \centering
    % \includegraphics[width=0.9\textwidth]{path/to/your/system_overview_diagram.png} % Replace with your actual diagram
    \fbox{\parbox[c][10cm][c]{0.9\textwidth}{\centering Placeholder for System Overview Diagram \\ (Raw Seismogram -> Conventional Processing -> Initial Picks -> CNN Refinement -> Final Event List)}}
    \caption{Conceptual overview of the Betelwise1 hybrid system architecture.}
    \label{fig:system_overview}
\end{figure}

\subsection{Stage 1: Conventional Pre-processing and Initial Picking}
This stage prepares the raw seismic data and identifies high-potential event candidates using a series of robust, tunable algorithms.

\subsubsection{Adaptive Frequency Band Estimation}
To enhance the SNR of potential seismic signals, the system first determines an optimal bandpass filter window for each seismogram. This is achieved by analyzing the signal's characteristics, either through power spectrum analysis to identify dominant frequencies associated with events, or through standard deviation analysis across different frequency bands. The parameters for this estimation (e.g., window lengths, frequency ranges) are tunable to adapt to different mission environments (e.g., Mars vs. Moon) or instrument responses.

\subsubsection{Bandpass Filtering}
Once the optimal frequency window is determined, a bandpass filter is applied to the seismogram. This step aims to remove out-of-band noise, including instrumental noise and microseisms, thereby isolating the frequency content most likely to contain seismic event energy.

\subsubsection{Outlier Removal}
Short-duration, high-amplitude noise spikes or glitches, which can be common in planetary seismic data due to various environmental or instrumental factors, are removed. This step is crucial to prevent such outliers from disproportionately affecting subsequent normalization and picking stages. A median-based despiking filter or a similar robust statistical method is employed.

\subsubsection{Data Normalization}
The filtered seismogram is then normalized (e.g., to zero mean and unit standard deviation, or min-max scaling) to ensure consistency across different data segments and amplitude ranges. This places the data on a consistent scale, which is beneficial for the STA/LTA analysis and subsequent CNN processing.

\subsubsection{STA/LTA Analysis}
The classic Short-Term Average/Long-Term Average (STA/LTA) algorithm \citep{allen1982automatic} is applied to the normalized data to detect potential seismic events. This involves calculating the ratio of the average signal amplitude in a short trailing window to that in a long trailing window. When this ratio exceeds a predefined threshold, a potential event onset is flagged. The lengths of the short and long windows, as well as the trigger/detrigger thresholds, are configurable parameters.

\subsubsection{Characteristic Function-Based Pick Filtering}
The picks generated by STA/LTA can sometimes include false positives due to non-seismic transients or complex noise. To refine these initial picks, a filter based on a "characteristic function" is applied. This function analyzes the waveform morphology around the STA/LTA pick, specifically examining features like the slope of the signal rise and fall, and the ratio of peak amplitude to pre-event noise. Picks that do not exhibit characteristics typical of genuine seismic onsets (e.g., an impulsive rise followed by a gradual decay) are filtered out. This step aims to significantly reduce the number of candidate events passed to the more computationally intensive CNN stage.

\subsection{Stage 2: CNN-based Event Refinement}
The candidate events that pass the conventional processing stage are further scrutinized by a lightweight Convolutional Neural Network (CNN).

\subsubsection{CNN Input Preparation}
For each candidate pick, a fixed-length 1D seismogram segment is extracted, centered around the pick time. In our current implementation, this segment has a shape of (5565, 1), corresponding to a specific duration at a given sampling rate.
In addition to the raw waveform segment, auxiliary inputs are provided to the CNN. These currently consist of the standard deviation of the seismic signal in windows immediately preceding and following the picked arrival time. These auxiliary features provide the CNN with additional context about the signal's statistical properties around the event.

\subsubsection{CNN Architecture}
The CNN is designed to be lightweight to minimize computational demands for on-board deployment. A typical architecture (Figure \ref{fig:cnn_architecture}) might consist of:
\begin{itemize}
    \item Several 1D convolutional layers with Rectified Linear Unit (ReLU) activation functions to extract hierarchical features from the waveform.
    \item Max-pooling layers for dimensionality reduction and to introduce some degree of translation invariance.
    \item One or more fully connected layers to integrate the learned features.
    \item A final output layer (e.g., with a sigmoid activation for binary classification) to predict the probability that the input segment represents a true seismic event versus noise.
\end{itemize}
The auxiliary inputs are typically concatenated with the flattened output of the convolutional/pooling layers before being fed into the fully connected layers.

\begin{figure}[H]
    \centering
    % \includegraphics[width=0.7\textwidth]{path/to/your/cnn_architecture_diagram.png} % Replace with your actual diagram
    \fbox{\parbox[c][8cm][c]{0.7\textwidth}{\centering Placeholder for CNN Architecture Diagram}}
    \caption{Conceptual architecture of the lightweight CNN used for event refinement. Input includes a 1D seismogram segment and auxiliary statistical features.}
    \label{fig:cnn_architecture}
\end{figure}

\subsubsection{CNN Training and Operation}
The CNN is trained offline using labeled datasets of seismic events and noise. During on-board operation, the pre-trained CNN takes the waveform segment and auxiliary features for each candidate pick and outputs a confidence score. Events exceeding a certain confidence threshold are considered true seismic events and prioritized for data downlink.

This hybrid approach allows Betelwise1 to leverage the efficiency and interpretability of conventional methods for initial data triage, while harnessing the pattern recognition power of CNNs for final, high-accuracy event classification.

\section{Experimental Setup}
To evaluate the performance of the Betelwise1 system, we designed a series of experiments using publicly available seismic datasets and established evaluation metrics.

\subsection{Datasets}
\begin{itemize}
    \item \textbf{Training Data:} The CNN component of Betelwise1 was trained on a diverse dataset of terrestrial earthquakes and noise segments obtained from the Incorporated Research Institutions for Seismology (IRIS) data management center. Events were selected to cover a range of magnitudes, epicentral distances, and noise conditions. For specific planetary applications, synthetic seismograms for Mars \citep{clinton2017marsquake} or the Moon could also be incorporated, or transfer learning from Earth data similar to \citet{civilini2021detecting} can be employed.
    \item \textbf{Testing Data:}
    \begin{itemize}
        \item \textbf{Lunar Data:} To compare with previous work and assess performance on real planetary data, we used data from the Apollo Passive Seismic Experiment (PSE) as utilized by \citet{civilini2021detecting}. This allows for an assessment of the system's ability to detect known moonquakes.
        \item \textbf{Martian Data (Simulated/Actual):} Performance was also (or will be) evaluated on seismic data from the NASA InSight mission to Mars \citep{lognonne2019seis, giardini2020seismicity}, focusing on its SEIS instrument. If real data with confirmed event catalogs is limited, high-fidelity synthetic Martian seismograms are used.
        \item \textbf{Earth Data (Holdout Set):} A separate holdout set of terrestrial earthquake data, not used during training, was used to establish baseline performance.
    \end{itemize}
    All data was pre-processed (e.g., de-trending, instrument response removal if necessary) and resampled to a consistent sampling rate suitable for the fixed input size of the CNN (e.g., corresponding to the 5565 samples). Event labels (P-arrivals, S-arrivals, noise) were obtained from existing catalogs or manual expert picking.
\end{itemize}

\subsection{Baselines for Comparison}
The performance of Betelwise1 was compared against:
\begin{enumerate}
    \item \textbf{Standard STA/LTA:} The conventional STA/LTA algorithm applied directly to the pre-processed data, with parameters optimized for each dataset.
    \item \textbf{Conventional Pipeline Only:} The full conventional pre-processing and picking pipeline of Betelwise1 (Section 3.1) without the final CNN refinement stage.
    \item \textbf{Spectrogram-based CNN (Conceptual):} While not fully re-implemented, conceptual comparisons are drawn with the performance characteristics of spectrogram-based CNNs like that of \citet{civilini2021detecting}, particularly regarding input data representation and potential computational trade-offs.
\end{enumerate}

\subsection{Evaluation Metrics}
The following metrics were used to quantify performance:
\begin{itemize}
    \item \textbf{Precision:} The fraction of detected events that are true events.
    $P = TP / (TP + FP)$
    \item \textbf{Recall (Sensitivity):} The fraction of true events that are correctly detected.
    $R = TP / (TP + FN)$
    \item \textbf{F1-Score:} The harmonic mean of precision and recall.
    $F1 = 2 \cdot (P \cdot R) / (P + R)$
    \item \textbf{False Positive Rate (FPR):} The fraction of non-events incorrectly classified as events.
    \item \textbf{Data Reduction Ratio:} The ratio of the original data volume to the volume of data identified for downlink by the system.
    \item \textbf{Computational Cost:} Inference time per unit of seismic data (e.g., per hour) and model size.
\end{itemize}
Where TP = True Positives, FP = False Positives, FN = False Negatives.

\subsection{Implementation Details}
The Betelwise1 system was implemented in Python. Seismic data processing utilized the ObsPy library \citep{obspy2010}. The CNN component was built and trained using TensorFlow/Keras. Experiments were run on [Specify computing hardware, e.g., NVIDIA GPU / CPU]. The parameters for conventional algorithms and CNN training (e.g., learning rate, batch size, epochs) were optimized through cross-validation on the training set.

\section{Results}
This section presents the preliminary results of the Betelwise1 system evaluation. (Note: These are illustrative results; replace with your actual findings.)

\subsection{Performance of Conventional Algorithms}
The conventional pre-processing and initial picking stage (Section 3.1) was first evaluated independently. On the [Specify Dataset, e.g., Apollo PSE lunar data], the adaptive bandpass filtering significantly improved the SNR of known events. The STA/LTA algorithm, when tuned, achieved a recall of [e.g., 75\%] but with a relatively high number of false positives. The application of the characteristic function-based filter successfully reduced the false positive count by [e.g., 40\%] while maintaining a recall of [e.g., 70\%] for clearly defined events. Table \ref{tab:conventional_results} summarizes these findings.

\begin{table}[H]
    \centering
    \caption{Performance of the Conventional Picking Stage on [Dataset Name].}
    \label{tab:conventional_results}
    \begin{tabular}{@{}lccc@{}}
        \toprule
        Method Stage                  & Precision & Recall & F1-Score \\ \midrule
        STA/LTA Only                  & [e.g., 0.50]   & [e.g., 0.75]  & [e.g., 0.60]  \\
        STA/LTA + Char. Func. Filter & [e.g., 0.65]   & [e.g., 0.70]  & [e.g., 0.67]  \\ \bottomrule
    \end{tabular}
\end{table}

\subsection{Performance of the Hybrid Betelwise1 System}
The full Betelwise1 system, incorporating CNN refinement, demonstrated a marked improvement in overall performance. When tested on the [Specify Dataset, e.g., Apollo PSE lunar data], the system achieved:
\begin{itemize}
    \item Precision: [e.g., 0.85]
    \item Recall: [e.g., 0.78]
    \item F1-Score: [e.g., 0.81]
\end{itemize}
Figure \ref{fig:pr_curve} shows the precision-recall curve for the Betelwise1 system compared to the conventional pipeline alone. The CNN refinement stage effectively filtered out many of the false positives identified by the conventional pickers, leading to higher precision with only a minor trade-off in recall for very low-amplitude events.

\begin{figure}[H]
    \centering
    % \includegraphics[width=0.7\textwidth]{path/to/your/pr_curve.png} % Replace with your PR curve
    \fbox{\parbox[c][8cm][c]{0.7\textwidth}{\centering Placeholder for Precision-Recall Curve}}
    \caption{Precision-Recall curve for Betelwise1 (hybrid) vs. Conventional Pipeline Only on [Dataset Name].}
    \label{fig:pr_curve}
\end{figure}

\subsection{Comparison with Baselines}
Compared to using only the standard STA/LTA algorithm, Betelwise1 showed a significant increase in F1-score by [e.g., 35\%]. The hybrid system also outperformed the standalone conventional pipeline of Betelwise1, primarily due to the CNN's superior ability to distinguish subtle events from noise that mimics seismic signals.

\subsection{Ablation Study: Impact of Auxiliary Features}
To assess the contribution of the auxiliary statistical features (standard deviations before/after arrival) to the CNN, an ablation study was performed. A version of the CNN was trained using only the 1D waveform segment as input. The inclusion of auxiliary features resulted in an F1-score improvement of [e.g., 5-7\%], suggesting they provide valuable contextual information to the network.

\subsection{Computational Efficiency and Data Reduction}
The lightweight CNN model has a size of [e.g., X MB] and an average inference time of [e.g., Y milliseconds] per candidate event on [target hardware, e.g., a Raspberry Pi / specific CPU]. The overall Betelwise1 system, by selectively identifying high-confidence events, achieved a data reduction ratio of [e.g., 50:1 to 100:1] compared to transmitting continuous raw data, while successfully capturing over [e.g., 75\%] of cataloged events on the test datasets. This demonstrates its potential for optimizing telemetry bandwidth.

\section{Discussion}
The preliminary results from the Betelwise1 system are promising, suggesting that a hybrid approach combining conventional seismic processing with lightweight CNN refinement can be an effective strategy for efficient planetary seismic event detection.

The multi-stage design of Betelwise1 offers several advantages. The conventional pre-processing and initial picking stage acts as an efficient first-pass filter, significantly reducing the number of candidate events that need to be processed by the more computationally intensive CNN. The adaptive bandpass filtering and the characteristic function-based pick filtering were shown to be crucial in improving the quality of candidate picks. This tiered approach is particularly relevant for resource-constrained lander environments.

The use of 1D waveform segments augmented with auxiliary statistical features as input to the CNN is a key differentiator from approaches like that of \citet{civilini2021detecting}, which primarily used 2D spectrograms. Our findings, particularly the improvement seen with auxiliary features, suggest that providing direct temporal information alongside statistical context can be a viable and potentially more computationally efficient alternative for on-board processing, as it avoids the need for on-the-fly spectrogram generation for every candidate window. However, a direct, rigorous comparison of computational load and performance against an optimized spectrogram-based CNN on the same hardware would be necessary to definitively establish superiority for all scenarios.

The performance on lunar data indicates that Betelwise1 can generalize to planetary environments, even if primarily trained on terrestrial data or with limited planetary examples. The observed F1-score of [e.g., 0.81] on the Apollo PSE data demonstrates a practical capability for identifying moonquakes. Further testing and adaptation for Martian seismic data, considering its unique noise characteristics and event types (e.g., high-frequency (HF) and low-frequency (LF) marsquakes \citep{giardini2020seismicity}), will be critical.

The data reduction ratios achieved by Betelwise1 are significant, highlighting its potential to alleviate telemetry bottlenecks in future planetary missions. By prioritizing high-confidence events, the system allows for more efficient use of limited downlink capacity, enabling the return of more scientifically valuable data.

\textbf{Limitations and Future Work:}
Despite these promising results, several limitations exist. The current CNN model, while lightweight, may still require further optimization for deployment on extremely power-constrained microcontrollers envisioned for some future missions. The system's performance is also dependent on the quality and representativeness of the training data; expanding the training set with more diverse planetary (real or synthetic) examples is crucial. The definition and tuning of the "characteristic function" could also be further refined, perhaps even learned using ML techniques.

Future work will focus on:
\begin{enumerate}
    \item Rigorous benchmarking on a wider range of planetary datasets, including more extensive testing on InSight Martian data.
    \item Investigating more advanced lightweight CNN architectures (e.g., MobileNets, SqueezeNets) or other ML models (e.g., Recurrent Neural Networks) for the refinement stage.
    \item Exploring unsupervised or self-supervised learning techniques to reduce reliance on large labeled datasets for initial training.
    \item Developing more sophisticated adaptive mechanisms for the conventional algorithms, potentially using feedback from the CNN.
    \item Implementing and testing the full Betelwise1 pipeline on representative low-power embedded hardware to accurately assess its on-board feasibility and power consumption.
    \item Conducting a direct comparative study against spectrogram-based CNNs on identical datasets and hardware to quantify trade-offs in performance and computational efficiency.
\end{enumerate}

\section{Conclusion}
This paper introduced Betelwise1, a novel hybrid system for efficient planetary seismic event detection. By integrating tunable conventional seismic processing algorithms with a lightweight Convolutional Neural Network operating on 1D waveforms and auxiliary statistical features, Betelwise1 aims to optimize data transmission from resource-constrained planetary missions. Preliminary results demonstrate the system's capability to accurately detect seismic events while significantly reducing data volume. The conventional stage effectively pre-filters candidate events, and the CNN provides robust final refinement.

The Betelwise1 approach offers a promising alternative and complement to existing methods, including those based purely on spectrogram analysis. Its focus on a hybrid architecture and direct waveform processing for the CNN component shows potential for efficient and reliable on-board analysis, particularly for future missions to Mars, the Moon, and other celestial bodies. Further development and validation on diverse planetary datasets and target hardware will be essential to fully realize its potential in advancing planetary seismology.

\section*{Acknowledgements}
(Optional: Thank individuals, institutions, or funding agencies. For Space Apps, you might thank NASA, the challenge organizers, mentors, etc.)
We thank the organizers of the NASA Space Apps Challenge for providing the platform for this research. We also acknowledge the providers of public seismic data, including IRIS DMC, and the teams behind the Apollo and InSight missions.

\bibliographystyle{plainnat}
\bibliography{references} % Assuming your .bib file is named references.bib

\end{document}